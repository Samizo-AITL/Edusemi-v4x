\documentclass[journal]{IEEEtran}

% ---- Safe, minimal packages (ASCII only) ----
\usepackage[T1]{fontenc}
\usepackage{lmodern}
\usepackage[utf8]{inputenc} % 入力はUTF-8だが本文はASCII推奨
\usepackage{graphicx}
\usepackage{amsmath,amssymb}
\usepackage{booktabs}
\usepackage{siunitx}
\usepackage[hidelinks]{hyperref}
\usepackage{cite}

\title{A Comparative Review of DRAM and FeRAM: Boundary and Perspectives}
\author{Shinichi~Samizo%
\thanks{S. Samizo is with Project Design Hub (Samizo-AITL), Japan. E-mail: \texttt{samizo-aitl@example.org}.}
}
\markboth{Draft for IEEE-style submission}{Samizo: DRAM vs FeRAM Comparative Review}

\begin{document}
\maketitle

\begin{abstract}
This note contrasts DRAM and FeRAM from device scaling to system use. We summarize speed, retention, endurance, and energy/bit, and discuss hybrid perspectives that may reshape the volatile--non-volatile boundary.
\end{abstract}

\begin{IEEEkeywords}
DRAM, FeRAM, FeFET, HfO$_2$, retention, endurance, scaling, memory hierarchy.
\end{IEEEkeywords}

% ----------------------------
\section{Introduction}
Memory hierarchies remain central to computing. DRAM dominates volatile memory owing to speed and scalability, while ferroelectric HfO$_2$-based FeRAM/FeFET have emerged as CMOS-compatible non-volatile options. Recent work highlights hybrid use-cases spanning IEDM and VLSI results.

% ----------------------------
\section{DRAM Technology and Scaling}
DRAM scaling progressed via high-k dielectrics and deep trench or stacked capacitors. Maintaining capacitance while suppressing leakage is a key challenge in sub-20\,nm nodes. Industry explores 3D architectures analogous to NAND flash.

\begin{figure}[!t]
\centering
\fbox{\parbox[c][4cm][c]{.9\linewidth}{\centering Placeholder for DRAM figure}}
\caption{Evolution of DRAM cell concepts (placeholder).}
\label{fig:dram_evolution}
\end{figure}

% ----------------------------
\section{FeRAM Technology and Advances}
Ferroelectric memories were pioneered in the 1950s; hafnia-based ferroelectrics revived research in the 2010s. FeFETs promise fast switching but face variability and endurance limits.

\begin{figure}[!t]
\centering
\fbox{\parbox[c][4cm][c]{.9\linewidth}{\centering Placeholder for FeRAM/FeFET figure}}
\caption{FeRAM / FeFET schematic (placeholder).}
\label{fig:feram}
\end{figure}

% ----------------------------
\section{Comparative Analysis: DRAM vs FeRAM}
Table~\ref{tab:comparison} summarizes representative metrics. DRAM provides ultra-high endurance ($\geq 10^{16}$ cycles) and low energy/bit, but retention is limited ($\sim$64 ms). FeRAM offers non-volatility ($>10^5$ s) and endurance of $10^{12}$--$10^{13}$ cycles, though with higher write energy.

\begin{table}[!t]
\centering
\caption{Comparison of DRAM and FeRAM metrics}
\label{tab:comparison}
\begin{tabular}{lcc}
\toprule
Metric & DRAM & FeRAM \\
\midrule
Speed & ns & ns--µs \\
Retention & $\sim$64 ms & $>10^5$ s \\
Endurance & $\geq 10^{16}$ & $10^{12}$--$10^{13}$ \\
Energy/bit & fJ & 100s fJ \\
Cell size & 6--8F$^2$ & 6--10F$^2$ \\
\bottomrule
\end{tabular}
\end{table}

% ----------------------------
\section{Hybrid Perspectives and Future Memory Hierarchies}
Hybrid designs combine DRAM speed with FeRAM persistence. Benefits: reduced refresh, fast checkpointing, crash consistency. Constraints: endurance gap, higher write energy, integration cost. System design: tiering, refresh co-optimization, FeRAM-backed logging.

\begin{figure}[!t]
\centering
\fbox{\parbox[c][4cm][c]{.9\linewidth}{\centering Placeholder for hybrid memory hierarchy}}
\caption{Hybrid memory hierarchy integrating DRAM and FeRAM (placeholder).}
\label{fig:hybrid_hierarchy}
\end{figure}

% ----------------------------
\section{Conclusion and Outlook}
DRAM will remain the backbone of volatile memory. FeRAM offers a CMOS-compatible non-volatile complement. Future work will bridge device-level improvements with system-level architecture for hybrid deployments.

% ----------------------------
% References
\nocite{*}
\bibliographystyle{IEEEtran}
\bibliography{references}

\end{document}
