\documentclass[journal]{IEEEtran}

% ---- Safe, minimal packages (ASCII only) ----
\usepackage[T1]{fontenc}
\usepackage{lmodern}
\usepackage[utf8]{inputenc} % 入力はUTF-8だが本文はASCII推奨
\usepackage{graphicx}
\usepackage{amsmath,amssymb}
\usepackage{booktabs}
\usepackage{siunitx}
\usepackage[hidelinks]{hyperref}
\usepackage{cite}

\title{A Comparative Review of DRAM and FeRAM: Boundary and Perspectives}
\author{Shinichi~Samizo%
\thanks{S. Samizo is with Project Design Hub (Samizo-AITL), Japan. 
E-mail: \texttt{samizo-aitl@example.org}.}
}
\markboth{Draft for IEEE-style submission}{Samizo: DRAM vs FeRAM Comparative Review}

\begin{document}
\maketitle

\begin{abstract}
This note contrasts DRAM and FeRAM from device scaling to system use. 
We summarize speed, retention, endurance, energy per bit, and perspectives 
for hybrid integration.
\end{abstract}

\begin{IEEEkeywords}
DRAM, FeRAM, FeFET, HfO2, retention, endurance, scaling, memory hierarchy.
\end{IEEEkeywords}

% -------------------------------
\section{Introduction}
Memory hierarchies remain central to computing. DRAM dominates volatile 
memory owing to speed and density, but faces scaling challenges as cell 
capacitors approach physical limits. FeRAM, based on ferroelectric hafnia, 
provides non-volatile operation with CMOS compatibility. This review 
compares DRAM and FeRAM, and discusses hybrid perspectives.

% -------------------------------
\section{DRAM Technology and Scaling}
DRAM scaling progressed via high-k dielectrics and deep trench or stacked 
capacitors. Maintaining capacitance while reducing leakage in sub-20 nm 
technology nodes is challenging \cite{choi2022}. 
Recent directions explore 3D DRAM architectures, analogous to NAND flash, 
by stacking capacitor arrays \cite{kim2021_dram}.

% Example figure (replace with real image if available)
\begin{figure}[!t]
\centering
\includegraphics[width=0.8\linewidth]{dram_example.png}
\caption{Evolution of DRAM cell concepts: planar, trench, stacked, and 3D.}
\label{fig:dram}
\end{figure}

% -------------------------------
\section{FeRAM Technology and Advances}
FeRAM based on HfO2 ferroelectricity enables fast, non-volatile switching 
\cite{boscke2011,mueller2012}. FeFET devices combine ferroelectric layers 
with MOS transistors, providing low-power non-volatile operation. 
Key issues remain: variability, endurance, and retention.

\begin{figure}[!t]
\centering
\includegraphics[width=0.8\linewidth]{feram_example.png}
\caption{Schematic concept of FeRAM/FeFET device.}
\label{fig:feram}
\end{figure}

% -------------------------------
\section{Comparative Analysis: DRAM vs FeRAM}
Table~\ref{tab:comparison} summarizes typical metrics. DRAM offers ultra-high 
endurance ($\geq 10^{16}$ cycles) and low energy per bit, but limited 
retention ($\sim$64 ms). FeRAM provides non-volatility with retention 
$>10^{5}$ s and endurance $10^{12}$--$10^{13}$ cycles, but higher write 
energy (hundreds of fJ/bit) \cite{noheda2023,martin2020}.

\begin{table}[!t]
\centering
\caption{Comparison of DRAM and FeRAM metrics}
\label{tab:comparison}
\begin{tabular}{lcc}
\toprule
Metric & DRAM & FeRAM \\
\midrule
Speed & ns & ns--us \\
Retention & $\sim$64 ms & $>10^{5}$ s \\
Endurance & $\geq 10^{16}$ & $10^{12}$--$10^{13}$ \\
Energy/bit & tens of fJ & hundreds of fJ \\
Cell size & 6--8F$^{2}$ & 6--10F$^{2}$ \\
\bottomrule
\end{tabular}
\end{table}

% -------------------------------
\section{Hybrid Perspectives}
Hybrid hierarchies aim to combine DRAM speed with FeRAM persistence. 
Placing FeRAM close to DRAM or memory controllers can reduce refresh, 
enable instant-on, and support fast checkpointing.

\begin{figure}[!t]
\centering
\includegraphics[width=0.85\linewidth]{hybrid_example.png}
\caption{Hybrid hierarchy concept: DRAM as working memory, FeRAM as 
near-memory store.}
\label{fig:hybrid}
\end{figure}

Benefits:
\begin{itemize}
\item Reduced refresh traffic and standby power.
\item Fast persistence with microsecond checkpoint.
\item Data resilience for metadata and buffers.
\end{itemize}

Constraints:
\begin{itemize}
\item Endurance and variability remain lower than DRAM refresh cycles.
\item Write energy higher than DRAM.
\item Integration cost at advanced CMOS nodes.
\end{itemize}

% -------------------------------
\section{Conclusion}
Comparative analysis suggests DRAM will continue as the high-speed volatile 
memory, while FeRAM provides non-volatile options with CMOS compatibility. 
Hybrid memory may redefine future system architectures.

% -------------------------------
\bibliographystyle{IEEEtran}
\bibliography{references}

\end{document}
