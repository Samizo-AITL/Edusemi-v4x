Memory hierarchies are critical to modern computing systems. DRAM remains the dominant volatile memory owing to its high speed, density, and scalability \cite{choi2022,kim2021_dram}. However, DRAM scaling faces fundamental challenges as the capacitor dimensions approach physical limits, particularly in sub-1z nodes \cite{kim2021_dram,iedm2023_dram}. 

In parallel, ferroelectric HfO$_2$-based memories (FeRAM/FeFET) have emerged as promising non-volatile candidates, combining CMOS compatibility with fast switching \cite{boscke2011,mueller2012,noheda2023}. This review consolidates results from IEDM, VLSI, and IEEE journals, while exploring hybrid perspectives that may reshape the volatile--non-volatile boundary.

The foundation of DRAM traces back to Dennard's seminal one-transistor/one-capacitor (1T1C) design in 1966 \cite{dennard1966}, whereas ferroelectric memories were pioneered in the 1950s \cite{scott1998}. Recent breakthroughs in ferroelectric hafnia \cite{boscke2011,mueller2012} revitalized interest in non-volatile memory research.
