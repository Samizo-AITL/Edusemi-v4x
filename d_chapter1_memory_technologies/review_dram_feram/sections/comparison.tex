Table~\ref{tab:comparison} summarizes representative metrics for DRAM and FeRAM, including speed, retention, endurance, energy per bit, and cell footprint. 

DRAM provides ultra-high endurance ($\geq 10^{16}$ cycles) and low energy per bit, but suffers from limited retention ($\sim$64 ms) and significant refresh overhead. In contrast, FeRAM offers non-volatility with retention exceeding $10^{5}$ s and endurance in the range of $10^{12}$--$10^{13}$ cycles, though its write energy is higher (hundreds of fJ/bit) \cite{noheda2023,martin2020}. 

Fig.~\ref{fig:svr} shows the speed--retention trade-off, while Fig.~\ref{fig:evs} highlights the relationship between write energy and speed. These comparative plots illustrate that DRAM is optimal for working memory, whereas FeRAM is better suited for low-power non-volatile applications.
