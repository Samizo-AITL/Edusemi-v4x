Table~\ref{tab:comparison} summarizes representative metrics (speed, retention, endurance, energy/bit, and cell footprint) for DRAM and FeRAM. 

DRAM provides ultra-high endurance ($\geq 10^{16}$ cycles) and low energy/bit, but suffers from limited retention ($\sim$64 ms) and high refresh overhead. FeRAM, by contrast, offers non-volatility with retention exceeding $10^5$ s and endurance of $10^{12}$--$10^{13}$ cycles, but its write energy is higher (hundreds of fJ/bit) \cite{noheda2023,martin2020}. 

Fig.~\ref{fig:svr} shows the speed–retention map, while Fig.~\ref{fig:evs} highlights the trade-off between write energy and speed. These comparative plots illustrate that DRAM is optimal for working memory, whereas FeRAM is suitable for low-power non-volatile applications.
