The discovery of ferroelectricity in HfO$_2$ thin films \cite{boscke2011} revitalized FeRAM research, enabling aggressive scaling and CMOS back-end compatibility. FeFET devices integrate ferroelectric layers directly into the gate stack, offering non-volatile operation with near-DRAM speed \cite{mueller2012,li2022}. 

Nevertheless, challenges persist in retention, fatigue, and variability, especially under scaled dimensions \cite{schenk2015,kim2020}. Recent progress has demonstrated endurance beyond $10^{12}$ cycles and enhanced retention stability in scaled FeFETs \cite{li2022}. Reliability issues such as TDDB under high-field operation remain important, as well as defect generation under stress \cite{park2020}.

Fig.~\ref{fig:fefet_structure} schematically shows a FeFET, where the ferroelectric layer is integrated with the MOSFET gate stack. This structure drastically reduces footprint compared with capacitor-based FeRAM, enabling high-density integration.
