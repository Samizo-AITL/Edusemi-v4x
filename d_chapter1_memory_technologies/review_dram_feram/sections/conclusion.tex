Comparative analysis suggests that DRAM will continue as the workhorse volatile memory, maintaining unmatched endurance and density. Meanwhile, FeRAM provides a compelling pathway toward embedded non-volatile solutions with CMOS compatibility \cite{noheda2023,martin2020}. 

The convergence of DRAM and FeRAM in hybrid hierarchies could redefine system memory, but scaling, reliability, and integration challenges remain open research frontiers. Future work should focus on bridging device-level improvements with architecture-level deployment, ensuring that ferroelectric memories can complement or even rival DRAM in mainstream applications.
