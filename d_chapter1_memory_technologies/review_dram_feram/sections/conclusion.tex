Comparative analysis suggests that DRAM will continue as the workhorse volatile memory, maintaining unmatched endurance and density. 
Meanwhile, FeRAM offers a compelling pathway toward embedded non-volatile solutions with competitive speed and CMOS process compatibility~\cite{noheda2023, martin2020}. 
The convergence of DRAM and FeRAM within hybrid hierarchies could redefine system memory, but scaling, reliability, and integration challenges remain open research frontiers. 
Future work should focus on bridging device-level improvements with architecture-level deployment, ensuring that emerging ferroelectric memories can complement or eventually rival DRAM in mainstream applications.
