Hybrid memory hierarchies have gained significant attention, aiming to exploit DRAM for high-speed operation together with FeRAM or other non-volatile memories \cite{samsung2021,itrs2022}. System-level perspectives envision FeRAM as near-memory storage or NVDIMM-like solutions, reducing refresh energy and enabling instant-on functionality. 

Such architectures may also support AI edge computing and in-memory acceleration, where FeFET-based devices provide low-power and non-volatile synaptic elements \cite{schaller2021}. Trade-offs in endurance, variability, and cost remain barriers to widespread adoption. Hybrid hierarchies may also integrate PCM, MRAM, and RRAM alongside FeRAM \cite{itrs2022}. 

Fig.~\ref{fig:hybrid_hierarchy} depicts a conceptual hierarchy integrating DRAM (working memory), FeRAM (near-memory storage), and SSD/HDD (bulk storage).
