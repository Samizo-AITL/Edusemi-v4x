DRAM technology has continually advanced through cell capacitor scaling, high-k dielectrics, and process innovations. 
Recent works highlight the difficulty of maintaining capacitance while suppressing leakage currents in deep sub-20 nm technologies \cite{choi2022}.
Furthermore, the industry is exploring 3D DRAM architectures to extend scaling, analogous to NAND flash, by stacking capacitor arrays \cite{kim2021_dram}.
Such approaches aim to overcome cell aspect ratio limits, though integration complexity and refresh overhead remain unresolved.

\begin{figure}[!t]
\centering
\begin{tikzpicture}[font=\footnotesize, >=Latex]
\node[draw, minimum width=1.8cm, minimum height=1.2cm] (planar) {Planar\\(1970s)};
\node[draw, minimum width=1.8cm, minimum height=1.2cm, right=1.2cm of planar] (trench) {Trench\\(1990s)};
\node[draw, minimum width=1.8cm, minimum height=1.2cm, right=1.2cm of trench] (hac) {High-Aspect\\(2000s)};
\node[draw, minimum width=1.8cm, minimum height=1.2cm, right=1.2cm of hac] (ddram) {3D DRAM?\\(2020s+)};
\draw[->] (planar) -- (trench);
\draw[->] (trench) -- (hac);
\draw[->] (hac) -- (ddram);
\node[below=0.2cm of planar] {MOS cap on Si};
\node[below=0.2cm of trench] {Deep trench cap};
\node[below=0.2cm of hac] {Tall fin/stack cap};
\node[below=0.2cm of ddram] {Stacked array concept};
\end{tikzpicture}
\caption{Evolution of DRAM cell structures and concepts.}
\label{fig:dram_evolution}
\end{figure}
